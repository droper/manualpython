%%%%%%%%%%%%%%%%%%%%%%%%%%%%%%%%%%%%%%%%%%%%%%%%%%%%%%%%%%%%%%%%%%%%%%%%%%%
%
% Clases de Postgresql
%
%%%%%%%%%%%%%%%%%%%%%%%%%%%%%%%%%%%%%%%%%%%%%%%%%%%%%%%%%%%%%%%%%%%%%%%%%%%

\documentclass[12pt,a4paper,oneside, openany]{book}
\textheight=26cm
\textwidth=18cm
\topmargin=-2cm
\oddsidemargin=-1cm
\parindent=0mm

\usepackage[utf8x]{inputenc}
\usepackage[spanish]{babel}
\usepackage{ucs}
\usepackage{amsmath,amsfonts,amssymb}
\usepackage{graphicx}
\usepackage{anyfontsize}
\usepackage{float}
\usepackage{textcomp}
%Paquete verbments para obtener bonito código fuente
\usepackage{verbments}

\fvset{frame=single,framerule=1pt}
\plset{language=python,texcl=true,style=vs,%
captionfont=\sffamily\color{white}}

\setcounter{chapter}{0}

\pagestyle{plain}


%--------------------------------------------------------------------------
%titulo
\title{ 
 \begin{center}
  {Python - Guia de clases}\\
 \end{center}
}

\author{Ing. Pedro Muñoz del Rio\\
        pmunoz@gmail.com}

\date{Lima, Perú}

\begin{document}

\renewcommand{\baselinestretch}{1.5} %espacio entre lineas, 2 es doble espacio
\renewcommand{\contentsname}{Índice}
\renewcommand{\listfigurename}{Lista de Gr\'aficos}
\renewcommand{\chaptername}{Cap\'itulo}
\renewcommand{\bibname}{Fuentes de Informaci\'on}
\renewcommand{\figurename}{Gr\'afico}
\renewcommand{\tablename}{Tabla}

 \maketitle %despliega el titulo
 
  \frontmatter  %Secciones del documento que van con numeros romanos en el indice 
 
   El curso para el cuál este documento ha sido creado es una introducción para programadores al lenguaje de programación Python y es parte de una serie de cursos cuyo fin es introducir al lector en el desarrollo de sofware mediante esta potente herramienta.\\

Los objetivos del curso son familiarizar al estudiante con los principales elementos del lenguaje y enseñar a pensar de forma ``pythonica'', de tal manera que pueda empezar a elaborar sus propios programas.\\

Al terminar las clases, el alumno debe conocer y comprender los componentes fundamentales del lenguaje de programación y estar en capacidad de pensar y resolver problemas utilizando Python.\\

La presente Guía es un complemento a las clases dictadas de forma presencial. Se ha redactado para servir como hoja de ruta en el transcurso de las sesiones y como estructura del contenido a dictarse. \\ %Resumen y Abstract
  
 \tableofcontents %despliega el índice
 \listoffigures   %Tabla de figuras
      
 \mainmatter  %Cuerpo del documento  

 %Cuerpo del documento 

 \chapter{Programar}
    \input que_es_python.tex

 \chapter{Estructuras de control}
    \input estructuras_control.tex
    
 \chapter{Tipos de datos estructurados}
 	\input tipos_datos_estructurados.tex
 	
 \chapter{Funciones y Módulos}
 	\input funciones.tex 		 	 	

 \chapter{Programación orientada a objetos}
 	\input poo.tex 
 
 \chapter{Excepciones}
    \input excepciones.tex
  
 \chapter{Decoradores}
    \input decoradores.tex

 \chapter{Persistencia de la data}
    \input persistencia.tex
    
 \chapter{Introducción a Django}
    \input introduccion_django.tex

 \chapter{Arquitectura de Django}
    \input arquitectura_django.tex

   \bibliographystyle{miestilo}
   \bibliography{bibliografia}

\end{document}